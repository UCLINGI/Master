\documentclass{eplDoc}



\newcommand{\docType}	{Exam}
\newcommand{\docDate}	{2012/06}
\newcommand{\docAuthor}	{François Pelsser}
\newcommand{\courseCode}{LINGI2263}
\newcommand{\courseName}{Computational Linguistic}

\begin{document}
\maketitle
\newpage

\section{This question is about the double articulation of the language}

\subsection{Define and explain the notion of double articulation in the language} 
\subsection{Illustrate this concept by identifying and describing the relevant linguistic units in the following sentence : "He drives dangerously"} 
\subsection{Using units of this sentence, explain the notion of allomorph and the process of morphological derivation} 


\section{We are intersted in the CYK parsing algorithm}
\subsection{You will find below an erroneous version of this algorithm.  Propose a correct version.  Note: the input/output formulation is the same as the one described in the course slides and the sequence of words to be parsed is represented in the same fashion.  However the pseudo-code description needs to be corrected.} 
\subsection{The CYK parsing algorithm requires the grammar to be in Chomsky Normal Form (CNF). Where is this property used in the corrected pseudo code that you are giving in Q 2.1?  What would be the problem of applying this algorithm to general CFG not in CNF?}

\section{This question is about grammars}
\subsection{List and explain the difference between dependency and constituency grammars. Discuss their advantages or drawbacks} 
\subsection{Explain two possibles approaches to integrateagreement rules in a grammar of constituents and discuss their advantages or drawbacks}

\section{In the context of speech synthesis.  The grapheme-to-phoneme conversion module produces a string of phonemes from a string of letters.  This phonetic production uses a structure learned from a training corpus made of triples (word, phonetic string, category)}
\subsection{What is the name of this structure? Explain the principle on with this structure relies.} 
\subsection{Why is this structure appropriate to the grapheme-to-phneme  conversion?  Hint think at the exception dictionary} 

\end{document}
